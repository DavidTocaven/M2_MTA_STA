\chapter{Modélisation et analyse des premières opérations de chaque pièce}
Maintenant que nous connaissons un modèle valide pour une opération ainsi que les temps nécessaires au déplacement du chariot sur l'axe vertical et horizontal, nous allons pouvoir commencer à modéliser le travail du \emph{STA} sur deux opérations.

Nous allons, dans un premier temps, effectuer une modélisation en RdP Temporels d'une commande de deux opérations suite à quoi, nous en effectuerons une analyse grâce à une version de \emph{TINA} identique que dans le chapitre \ref{chap:realisationUneOperation}. Nous utiliserons cette analyse pour déterminer les intervalles d'attentes et le meilleur ordonnancement possible pour ne pas déclencher l'alarme.

\section{Réseau de PETRI Temporels de commande de deux opérations}

